\section{Curve Sketching and Related Concepts}

\subsection{Concavity}

\begin{itemize}
  \item Concave upwards (`Convex'): $f''(x) > 0$
  \item Concave downwards (`Concave'): $f''(x) < 0$
\end{itemize}

\subsection{Local maximum / minimum}

To `sketch' the curve (more officially `graph') of a function $f$, we need, among other things, to find `local maximum / minimum points'.

To find them, we can use \textbf{first derivative test} or \textbf{second derivative test}.

\subsubsection{First Derivative Test}

A point $p$ is called the local minimum (maximum) of $f(x)$ if

\begin{enumerate}
  \item $f'(p - \epsilon) < 0$ ($f'(p - \epsilon) > 0$) for small positive $\epsilon$
  \item $f'(p + \epsilon) > 0$ ($f'(p + \epsilon) < 0$) for small positive $\epsilon$
  \item $f$ is continuous at $x = p$
\end{enumerate}

This can be proved by integration or Mean Value Theorem.

\subsubsection{Second Derivative Test}

A point $p$ is called the local minimum (maximum) of $f(x)$ if

\begin{enumerate}
  \item \textbf{Assumption:} $f''(x)$ is continuous near $p$
  \item $f'(p) = 0$
  \item $f''(p) < 0$ ($f''(p) > 0$)
\end{enumerate}

This implies that $f'(p)$ is strictly decreasing (strictly increasing), and hence by the First Derivative Test, $p$ is the local maximum (local minimum) point.

\subsection{Inflection Point}

The point at which $f''(x)$ changes sign, or the concavity of $f(x)$ changes.

\section{Integration Theory}

There are two main concepts in integration:

\begin{itemize}
  \item Indefinite Integration (not about area)
  \item Definite Integration (about area)
\end{itemize}

\subsection{Indefinite Integration}

What is $\int f(x) dx$? How is it related to derivatives?

If we let $F(x) = \int f(x) dx$, then we know that $$\frac{d}{dx} F(x) = f(x)$$

This equation is also an example of a \textbf{first order ordinary differential equation}, which is an equation of the form $$ \frac{dF(x)}{dx} = f(x) $$ ($f(x)$ is given, and $F(x)$ is the unknown.)

\begin{quote}
  Example: $$ \frac{dF(x)}{dx} = x \implies F(x) = \frac{1}{2}x^2 + C $$
  
  That $C$ is a constant can be proved with Mean Value Theorem.
\end{quote}