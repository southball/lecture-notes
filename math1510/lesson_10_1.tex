\subsection{Integration Problem Examples}

\begin{equation*}
\begin{split}
    \int 3x^2 \arctan (x^3) dx
    &= \int \arctan (x^3) d(x^3)\\
    &= x^3 \arctan (x^3) - \int x^3 \cdot \frac{3x^2}{1 + x^3} dx
\end{split}
\end{equation*}

\subsection{Half Angle t-Substitution}

How do we evaluate the following integral?

$$ \int \frac{1}{1 - \sin x + \cos x} dx $$

Or, actually, any rational functions of $\sin x$ and $\cos x$ like $\frac{\sin x - 3 \cos x}{2 \sin x + 5 \cos x + 1}$?

We can use the technique t-substitution, but letting $t = \tan \frac{x}{2}$. Then we can express $\sin x, \cos x$ and $dx$ in terms of $t$.

\begin{flalign*}
\begin{split}
    \sin {x} 
    &= 2 \sin \frac{x}{2} \cos \frac{x}{3}\\
    &= 2 \cdot \frac{t}{\sqrt{1 + t^2}} \cdot \frac{1}{\sqrt{1 + t^2}}\\
    &= \frac{2t}{1 + t^2}
\end{split}&&
\end{flalign*}

\begin{flalign*}
\begin{split}
    \cos {x}
    &= \cos^2 {x} - \sin^2 {x}\\
    &= (\frac{1}{\sqrt{1 + t^2}})^2 - (\frac{t}{\sqrt{1 + t^2}})^2\\
    &= \frac{1 - t^2}{1 + t^2}
\end{split}&&
\end{flalign*}

\begin{flalign*}
\begin{split}
    dt
    &= d(\tan \frac{x}{2})\\
    &= \frac{1}{2} \sec^2 \frac{x}{2} dx\\
    &= \frac{1}{2} (1 + t^2) dx
\end{split}&&
\end{flalign*}

\begin{flalign*}
dx = \frac{2dt}{1 + t^2}&&
\end{flalign*}

Therefore

\begin{flalign*}
\begin{cases}
    \sin x = \frac{2t}{1 + t^2}\\
    \cos x = \frac{1 - t^2}{1 + t^2}\\
    dx = \frac{2dt}{1 + t^2}
\end{cases}&&
\end{flalign*}

\subsection{Trigonometric Substitution Revisited}

Let us consider integrals of the form $\int_{x=-R}^{x=R} \sqrt{R^2 - x^2} dx$. Let us compute the integral, taking into the account the domain of the integrands (i.e. $\sqrt{R^2 - x^2}$).

\textit{Note that the integrand must be continuous and the domain must be closed.}\\

\hrule

Let $x = R \sin \theta$.

\begin{quote}
    What is the domain of $\theta$?
    
    $$x = -R \implies R = R \sin \theta \implies \theta = -\frac{\pi}{2}$$
    $$x = R \implies R = R \sin \theta \implies \theta = \frac{\pi}{2}$$
    
    In the two equations above, there are many solutions. The `branch' (domain, term from complex function theory) $[-\frac{pi}{2}, \frac{pi}{2}]$ was chosen.
\end{quote}

Then, $dx = R \sin \theta d\theta$.

Finally,

\begin{equation*}
\begin{split}
    \int_{-R}^R \sqrt{R^2 - x^2} dx
    &= \int_{\theta = -\frac{pi}{2}}^{\theta = \frac{\pi}{2}} \sqrt{R^2 - R^2 \sin^2 \theta} \cdot R \cos \theta d\theta\\
    &= \int_{\theta = -\frac{pi}{2}}^{\theta = \frac{\pi}{2}} |R \cos \theta| \cdot R \cos \theta d\theta\\
    &= \int R^2 \cos^2 \theta\ (\text{for the branch } [-\frac{\pi}{2}, \frac{\pi}{2}] \text{, } \cos \theta \geq 0)
\end{split}
\end{equation*}

Also, for the branch chosen, we have $\frac{d\sin\theta}{d\theta} \geq 0$ and $\cos\theta \geq 0$.

\begin{quote}
    Interestingly, the proof above is actually a circular proof. When we compute $dx$, we used the fact that $\frac{d\sin\theta}{d\theta} = \cos\theta$, which relies on the fact that $\lim_{h\to 0} \frac{\sin h}{h} = 1$, which is derived from the area of circle $A = \frac{1}{2}r^2\theta$.
\end{quote}