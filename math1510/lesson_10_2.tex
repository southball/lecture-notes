\subsection{Integration by Rationalization}

\textbf{Problem:} Evaluate $\int \frac{\sqrt{x+1}}{x} dx$.

\textbf{Solution:} Let $u = \sqrt{x + 1}$.

Then $u^2 = x + 1 \implies 2u du = dx$.

\begin{equation*}
\begin{split}
    \int \frac{\sqrt{x + 1}}{x}
    &= \int \frac{u}{u^2 - 1} \cdot 2u du\\
    &= \int \frac{2}{u^2 - 1} du
\end{split}
\end{equation*}

Which is a rational function.

This can also be used to solve integration problems like $\int \sqrt{\frac{ax + b}{cx + d}} dx$.

\section{Riemann Sum}

\subsection{Introduction}

\textbf{Goal:} To compute the `area' (signed area) `under' the curve $y = f(x)$, for $a \leq x \leq b$.

\textbf{Idea:} To copy the idea of Archimedes of calculating area of circle. To do this, put `suitable' rectangles under the curve to approximate the area, and increase the number of rectangles so the sum of area of rectangles gets close to the area under the curve.

There are several ways to choose the rectangles. For example, the \textbf{lower sum} can be used.

\subsection{An Example of Riemann Sum}

Consider the curve $y = x$, and $-1 \leq x \leq 1$. Let $n = 4$. Then

$$ x_0 = -1, x_1 = -0.5, x_2 = 0, x_3 = 0.5, x_4 = 1 $$

is one method of choosing $x$.

Using \textbf{lower sum}, we have

\begin{equation*}
\begin{split}
    \textbf{Area}
    &= (-1) \cdot 0.5 + (-0.5) \cdot 0.5 + 0 \cdot 0.5 + 0.5 \cdot 0.5\\
    &= -0.5
\end{split}
\end{equation*}

Using \textbf{midpoint sum}, we have

\begin{equation*}
\begin{split}
    \textbf{Area}
    &= (-0.75) \cdot 0.5 + (-0.25) \cdot 0.5 + 0.25 \cdot 0.5 + 0.75 \cdot 0.5\\
    &= 0
\end{split}
\end{equation*}

The area under the curve can be approximated as follows:

$$ \int_{x=a}^{x=b} f(x) dx = \lim_{n\to\infty} \sum_{i=1}^n f(\xi_i) \cdot (x_i - x_{i-1}) = \lim_{n\to\infty} \sum_{i=1}^n f(\xi_i) \Delta x_i $$

where $\xi_i \in [x_{i-1}, x_i]$. Then it can be proved that any $\xi_i$ can be chosen without affecting the limit.

\subsection{Condition for Integrable Functions}

\begin{enumerate}
    \item The subdivision of $[a, b]$ into $n$ subintervals $[x_0, x_1], [x_1, x_2], \dots, [x_{n-1}, x_n]$ must satisfy the condition that $|\Delta x_i| \to 0$ as $n \to \infty$.
    
    \textbf{Why?} In order to avoid the situation like ($a = 0, b = 1$) $x_0 = 0, x_i = \frac{1}{2^i}$ for $i > 0$.
    
    \item $f(x)$ needs to be a continuous function on $[a, b]$.
\end{enumerate}

If condition 1 and 2 are satisfied, then it can be proved (using epsilon-delta definition) that $\lim_{n\to\infty} \sum_{i=1}^n f(\xi_i) \Delta x_i$ exists (and is finite), and is defined as the area under the curve.

This limit is denoted (given the notation) $\int_a^b f(x) dx$.

\subsection{Rough Idea of Proof}

Bound (control, find the upper and lower limit) $f(\xi_i)$ from above and below. 

\textrightarrow\ find $m_i, M_i$ such that $m_i \leq f(\xi_i) \leq M_i$. $m_i$ and $M_i$ are the \textbf{absolute minimum} and \textbf{absolute maximum} of $f(x)$ in $[x_{i-1}, x_i]$ respectively.

\begin{quote}
    As we are consider the \textbf{closed} interval $[x_{i-1}, x_i]$, the absolute minimum and maximum must exist.
\end{quote}

Therefore $m_i \Delta x_i \leq f(\xi_i) \Delta x_i \leq M_i \Delta x_i$.

\begin{quote}
    We wish that as $n \to \infty$, $\lim \sum_{i=1}^n m_i \Delta x_i = \lim \sum_{i=1}^n M_i \Delta x_i$.

    Then by sandwich theorem,
    
    $$ \lim_{n\to\infty} \sum_{i=1}^n m_i \Delta x_i = \lim_{n\to\infty} \sum_{i=1}^n f(\xi_i) \Delta x_i = \lim_{n\to\infty} \sum_{i=1}^n M_i \Delta x_i $$
    
    and it doesn't matter how we choose $\xi_i$ as long as $\xi_i \in [x_{i-1}, x_i]$.
\end{quote}

$\lim_{n\to\infty} \sum_{i=1}^n m_i \Delta x_i$ is named the \textbf{lower sum} and $\lim_{n\to\infty} \sum_{i=1}^n M_i \Delta x_i$ is denoted the \textbf{upper sum}. They are also called the \textbf{Darboux sums}.

Instead of finding out the upper sum / the lower sum, we estimate $\sum_{i=1}^n (M_i - m_i) \Delta x_i$.

\begin{quote}
    $M_i - m_i$ measures the `change' of $f(x)$ when $x_{i-1} \leq x \leq x_i$.
    
    If $f$ is continuous, it can be proved that $M_i - m_i$ can be made arbitrary small.
\end{quote}

\textbf{Applications:} this can be used to find the normal distribution table.

\subsection{Another Example of Riemann Sum}

Consider the function $y = f(x) = x$.

\textbf{Step 1:} let $x_0 = 0, x_1 = \frac{1}{n}, \dots, x_i = \frac{i}{n}, x_n = 1$.\\
This way of partitioning $[a, b]$ is called \textbf{equipartition}.

\textbf{Step 2:} $\Delta x_i = x_i - x_{i - 1} = \frac{i}{n} - \frac{i - 1}{n} = \frac{1}{n}$.

\textbf{Step 3:} We will choose the upper sum for this problem.

\begin{equation*}
\begin{split}
    \text{Upper sum}
    &= U_n\\
    &= \sum_{i=1}^n \frac{1}{n} \cdot {i}{n}\\
    &= \frac{1}{n^2} \sum_{i=1}^n i\\
    &= \frac{1}{n^2} \cdot \frac{n(n+1)}{2}\\
    &= \frac{n + 1}{2n}\\
    &= \frac{1}{2} (1 + \frac{1}{n})
\end{split}
\end{equation*}

\textbf{Step 4:} 
$$\lim_{n\to\infty} U_n = \frac{1}{2} = \int_0^1 x dx = \left[\frac{x^2}{2}\right]_0^1 $$

\textbf{Remark:} if $f(x) = x^\alpha$ for $\alpha > 1$, then equipartition does not work well.

\textbf{Note:} Mathematician \textit{Otto Toeplitz}'s approach of calculus. 

\section{Fundamental Theorem of Calculus}

\subsection{An Example}

Consider the problem

$$ \int_0^1 x dx $$

First, we find $\int x dx = \frac{x^2}{2} + C$.\\
Then, let $F(x) := \frac{x^2}{2} + C$.\\
Finally, compute $F(1) - F(0) = \int_0^1 x dx$.

How do we prove that this is correct?

Consider $f(x): [a, b] \to \mathbb{R}$, and let $x \in (a, b)$. Then we notice that the area of the curve $y = f(x)$ between $x$ and $x + \Delta x \approx f(x) \cdot \Delta x$.

But the area of the small rectangle is also equal to $\int_a^{x+\Delta x} f(t) dt - \int_a^x f(t) dt$, which is equal to $\int_x^{x+\Delta x} f(t) dt$.

\begin{quote}
    Or, by algebra,
    \begin{equation*}
    \begin{split}
        & \int_a^{x + \Delta x} f(t) dt - \int_a^x f(t) dt\\
        =& \int_a^{x + \Delta x} f(t) dt + \int_x^a f(t) dt\\
        =& \int_x^{x + \Delta x} f(t) dt
    \end{split}
    \end{equation*}
\end{quote}

where we have used

\begin{enumerate}
    \item $\int_a^b f(x) dx = -\int_b^a f(x) dx$ (from area of rectangle)
    \item $\int_a^b f(x) dx = \int_a^c f(x) dx + \int_c^b f(x)dx$
\end{enumerate}

But then because $\int_a^{x+\Delta x} f(t) dt = G(x + \Delta x)$, and $\int_a^x f(t) dt = G(x)$, where $G(x) := \int_a^x f(t) dt$,\\
so $\int_a^{x + \Delta x} f(t) dt - \int_a^x f(t) dt = G(x + \Delta x) - G(x)$.

Dividing $\Delta x$ from both sides, we have

$$ \frac{1}{\Delta x} \int_x^{x+\Delta x} f(t) dt \approx \frac{f(x) \Delta x}{\Delta x} = f(x) $$

Also

$$ \lim_{\Delta x \to 0} \frac{G(x + \Delta x) - G(x)}{\Delta x} \approx f(x) $$

We can use the \textbf{Integral Mean Value Theorem} to prove the two statements above more formally.

\begin{quote}
    The \textbf{Integral Mean Value Theorem} states that if $f$ is continuous in $[a, b]$, then there exists $\xi$ in $[a, b]$ such that
    
    $$ \int_a^b f(x) dx = f(\xi) (b - a) $$
    
    or, equivalently,
    
    $$ f(\xi) = \frac{1}{b - a} \int_a^b f(x) dx $$
\end{quote}

Using this theorem, we have

$$ \int_x^{x + \Delta x} f(t) dt = f(\xi) \cdot \Delta x $$

for some $\xi \in [x, x + \Delta x]$. Then,

$$ \lim_{\Delta x \to 0} \frac{\int_x^{x+\Delta x} f(t) dt}{\Delta x} = \lim_{\Delta x \to 0} f(\xi) $$

As $x \leq \xi \leq x + \Delta x$, by sandwich theorem, $\xi \to x$.

\textbf{Remark:} $\int_a^b x dx$

\textbf{Step 1:} $\int x dx = \frac{x^2}{2} + C$, i.e. find $G(x)$ such that $G'(x) = x$.

\textbf{Step 2:} If we can find the function $G(x)$, then $\int_a^b x dx = G(b) - G(a)$.