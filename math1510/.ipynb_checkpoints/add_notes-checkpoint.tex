\documentclass{article}
\usepackage{import}
\usepackage[utf8]{inputenc}
\usepackage{graphicx}
\usepackage[a5paper]{geometry}
\usepackage{amsfonts}
\usepackage{amsmath}

\title{MATH1510 Additional Notes}
\author{}
\date{}

\setlength\parindent{0em}
\setlength\parskip{1em}

\begin{document}

\section{Coursework 3}

\subsubsection{Problem 2}

$$
f(x) = \begin{cases}
  \frac{x^2-4}{x-2} & \text{if } x \not = 2\\
  4 & \text{if } x = 2
\end{cases},\ 
g(x) = \begin{cases}
  x & \text{if } x < 4\\
  x^2 - 3x & \text{if } x \geq 4
\end{cases}
$$

Show that $g(f(x))$ is continuous at $x = 0$.

Two approaches to the problem:

\begin{enumerate}
  \item Find some theorem saying that if $f(x)$ is continuous somewhere and $g(x)$ is continuous somewhere, then $g(f(x))$ is continuous.
  \item Start with scratch: simplify $g(f(x))$.
\end{enumerate}

\subsection{Problem 4}

Given that $$f(x + y) = f(x)f(y)$$, show $f'(x)$ exists if $f'(0)$ exists.

Difference quotient\\
$ = \frac{f(x + h) - f(x)}{h} (h \not = 0) $\\
$ = \frac{f(x)f(h) - f(x)}{h} $\\
$ = f(x) \cdot \frac{f(h) - 1}{h} $

Given that $f'(0)$ exists, we know that $lim_{h \to 0} \frac{f(h) - f(0)}{h}$ is defined.

We notice that $\frac{f(h) - f(0)}{h}$ does not look the same as $\frac{f(h) - 1}{h}$. Hence we want to prove that $f(0) = 1$.

\end{document}
