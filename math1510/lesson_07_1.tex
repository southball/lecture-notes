\section{Section of Taylor's Theorem}

\subsection{Estimation of Taylor's Theorem}

For all practical purposes, we're only interested in "truncated Taylor series" or Taylor polynomial of degree (order) $n$.

$$ TP_n(x, c) = f(c) + \sum_{i=1}^n \frac{f^{(i)}(c)}{i!} (x - c)^i $$

\subsection{Limitation of Taylor Polynomial}

Some functions cannot be represented by Taylor Polynomial. For example,

$$ f(x) = \begin{cases}

0 & x \leq 0\\
e^{-\frac{1}{x}} & x > 0

\end{cases} $$

$f^{(n)}$ for all $n > 0$ exists, and $f^{(n)}(0) = 0$ for all $n \geq 1$.

Consider the Taylor Polynomial

$$ f(c) + \sum_{i=1}^n \frac{f^{(i)}(c)}{i!}(x - c)^i $$

which is equal to 0.

We can also consider the similar function

$$ 
  g(x) = \begin{cases}
    e^{-\frac{1}{x^2}} & x \not = 0\\
    0 & x = 0
  \end{cases}
$$

which has similar properties.

\subsection{Analytic Functions}

If 

$$ f(x) = f(c) + \sum_{i=1}^\infty \frac{f^{(i)}(c)}{i!} (x-c)^i $$

then we say $f(x)$ is analytic at $x = c$.

For example, $\sin(x), \cos(x), \ln(1+x), e^x$ are all analytic functions.

\subsection{Estimation with Taylor Series}

Suppse we want to calculate the value of $e$. How to do it if the allowed error is $0.01$?

\textbf{Solution:} use truncated Taylor Series entered at some easy-to-calculate point $c$. \textrightarrow How to truncate, or how to choose $n$?

Assume for simplicity that $e < 3$. Choose the function $f(x) = e^x$ and $c = 0$. Then we have

$$ e = f(1) = f(0) + \frac{f'(0)}{1!} (x - 0)^1 + \dots + \frac{f^{(n)}(0)}{n!}(x - 0)^n + \frac{f^{(n+1)}(\xi)}{(n+1)!}(x - 0)^{(n+1)} $$

which is equal to

$$ 1 + \frac{x}{1!} + \dots + \frac{x^n}{n!} + \frac{e^\xi \cdot x^{(n + 1)}}{(n + 1)!} $$

and we have $0 < \xi < x$, i.e. $0 < \xi < 1$.

Hence we need to control the error term, so that it is $\leq 0.01$, i.e. $ | \frac{e^\xi}{(n + 1)!} | \leq 0.01 $

Then we need to find an ``n'' satisfying this inequality. We cannot solve the equation easily, so we solve $ | \frac{3}{(n+1)!} | \leq 0.01 $ instead.

Hence, we have $ n = 5 $.

$$ e^1 = 1 + 1 + \frac{1}{2!} + \dots + \frac{1}{5!} $$ approximates $e^1$ with error $< 0.01$.